% Generated by Sphinx.
\def\sphinxdocclass{report}
\documentclass[letterpaper,10pt,english]{sphinxmanual}
\usepackage[utf8]{inputenc}
\DeclareUnicodeCharacter{00A0}{\nobreakspace}
\usepackage{cmap}
\usepackage[T1]{fontenc}
\usepackage{amsfonts}
\usepackage{babel}
\usepackage{times}
\usepackage[Bjarne]{fncychap}
\usepackage{longtable}
\usepackage{sphinx}
\usepackage{multirow}
\usepackage{eqparbox}


\addto\captionsenglish{\renewcommand{\figurename}{Fig. }}
\addto\captionsenglish{\renewcommand{\tablename}{Table }}
\SetupFloatingEnvironment{literal-block}{name=Listing }



\title{MAOOAM Documentation}
\date{February 19, 2018}
\release{}
\author{Maxime Tondeur, Jonathan Demayer}
\newcommand{\sphinxlogo}{}
\renewcommand{\releasename}{Release}
\setcounter{tocdepth}{1}
\makeindex

\makeatletter
\def\PYG@reset{\let\PYG@it=\relax \let\PYG@bf=\relax%
    \let\PYG@ul=\relax \let\PYG@tc=\relax%
    \let\PYG@bc=\relax \let\PYG@ff=\relax}
\def\PYG@tok#1{\csname PYG@tok@#1\endcsname}
\def\PYG@toks#1+{\ifx\relax#1\empty\else%
    \PYG@tok{#1}\expandafter\PYG@toks\fi}
\def\PYG@do#1{\PYG@bc{\PYG@tc{\PYG@ul{%
    \PYG@it{\PYG@bf{\PYG@ff{#1}}}}}}}
\def\PYG#1#2{\PYG@reset\PYG@toks#1+\relax+\PYG@do{#2}}

\expandafter\def\csname PYG@tok@gd\endcsname{\def\PYG@tc##1{\textcolor[rgb]{0.63,0.00,0.00}{##1}}}
\expandafter\def\csname PYG@tok@gu\endcsname{\let\PYG@bf=\textbf\def\PYG@tc##1{\textcolor[rgb]{0.50,0.00,0.50}{##1}}}
\expandafter\def\csname PYG@tok@gt\endcsname{\def\PYG@tc##1{\textcolor[rgb]{0.00,0.27,0.87}{##1}}}
\expandafter\def\csname PYG@tok@gs\endcsname{\let\PYG@bf=\textbf}
\expandafter\def\csname PYG@tok@gr\endcsname{\def\PYG@tc##1{\textcolor[rgb]{1.00,0.00,0.00}{##1}}}
\expandafter\def\csname PYG@tok@cm\endcsname{\let\PYG@it=\textit\def\PYG@tc##1{\textcolor[rgb]{0.25,0.50,0.56}{##1}}}
\expandafter\def\csname PYG@tok@vg\endcsname{\def\PYG@tc##1{\textcolor[rgb]{0.73,0.38,0.84}{##1}}}
\expandafter\def\csname PYG@tok@vi\endcsname{\def\PYG@tc##1{\textcolor[rgb]{0.73,0.38,0.84}{##1}}}
\expandafter\def\csname PYG@tok@mh\endcsname{\def\PYG@tc##1{\textcolor[rgb]{0.13,0.50,0.31}{##1}}}
\expandafter\def\csname PYG@tok@cs\endcsname{\def\PYG@tc##1{\textcolor[rgb]{0.25,0.50,0.56}{##1}}\def\PYG@bc##1{\setlength{\fboxsep}{0pt}\colorbox[rgb]{1.00,0.94,0.94}{\strut ##1}}}
\expandafter\def\csname PYG@tok@ge\endcsname{\let\PYG@it=\textit}
\expandafter\def\csname PYG@tok@vc\endcsname{\def\PYG@tc##1{\textcolor[rgb]{0.73,0.38,0.84}{##1}}}
\expandafter\def\csname PYG@tok@il\endcsname{\def\PYG@tc##1{\textcolor[rgb]{0.13,0.50,0.31}{##1}}}
\expandafter\def\csname PYG@tok@go\endcsname{\def\PYG@tc##1{\textcolor[rgb]{0.20,0.20,0.20}{##1}}}
\expandafter\def\csname PYG@tok@cp\endcsname{\def\PYG@tc##1{\textcolor[rgb]{0.00,0.44,0.13}{##1}}}
\expandafter\def\csname PYG@tok@gi\endcsname{\def\PYG@tc##1{\textcolor[rgb]{0.00,0.63,0.00}{##1}}}
\expandafter\def\csname PYG@tok@gh\endcsname{\let\PYG@bf=\textbf\def\PYG@tc##1{\textcolor[rgb]{0.00,0.00,0.50}{##1}}}
\expandafter\def\csname PYG@tok@ni\endcsname{\let\PYG@bf=\textbf\def\PYG@tc##1{\textcolor[rgb]{0.84,0.33,0.22}{##1}}}
\expandafter\def\csname PYG@tok@nl\endcsname{\let\PYG@bf=\textbf\def\PYG@tc##1{\textcolor[rgb]{0.00,0.13,0.44}{##1}}}
\expandafter\def\csname PYG@tok@nn\endcsname{\let\PYG@bf=\textbf\def\PYG@tc##1{\textcolor[rgb]{0.05,0.52,0.71}{##1}}}
\expandafter\def\csname PYG@tok@no\endcsname{\def\PYG@tc##1{\textcolor[rgb]{0.38,0.68,0.84}{##1}}}
\expandafter\def\csname PYG@tok@na\endcsname{\def\PYG@tc##1{\textcolor[rgb]{0.25,0.44,0.63}{##1}}}
\expandafter\def\csname PYG@tok@nb\endcsname{\def\PYG@tc##1{\textcolor[rgb]{0.00,0.44,0.13}{##1}}}
\expandafter\def\csname PYG@tok@nc\endcsname{\let\PYG@bf=\textbf\def\PYG@tc##1{\textcolor[rgb]{0.05,0.52,0.71}{##1}}}
\expandafter\def\csname PYG@tok@nd\endcsname{\let\PYG@bf=\textbf\def\PYG@tc##1{\textcolor[rgb]{0.33,0.33,0.33}{##1}}}
\expandafter\def\csname PYG@tok@ne\endcsname{\def\PYG@tc##1{\textcolor[rgb]{0.00,0.44,0.13}{##1}}}
\expandafter\def\csname PYG@tok@nf\endcsname{\def\PYG@tc##1{\textcolor[rgb]{0.02,0.16,0.49}{##1}}}
\expandafter\def\csname PYG@tok@si\endcsname{\let\PYG@it=\textit\def\PYG@tc##1{\textcolor[rgb]{0.44,0.63,0.82}{##1}}}
\expandafter\def\csname PYG@tok@s2\endcsname{\def\PYG@tc##1{\textcolor[rgb]{0.25,0.44,0.63}{##1}}}
\expandafter\def\csname PYG@tok@nt\endcsname{\let\PYG@bf=\textbf\def\PYG@tc##1{\textcolor[rgb]{0.02,0.16,0.45}{##1}}}
\expandafter\def\csname PYG@tok@nv\endcsname{\def\PYG@tc##1{\textcolor[rgb]{0.73,0.38,0.84}{##1}}}
\expandafter\def\csname PYG@tok@s1\endcsname{\def\PYG@tc##1{\textcolor[rgb]{0.25,0.44,0.63}{##1}}}
\expandafter\def\csname PYG@tok@ch\endcsname{\let\PYG@it=\textit\def\PYG@tc##1{\textcolor[rgb]{0.25,0.50,0.56}{##1}}}
\expandafter\def\csname PYG@tok@m\endcsname{\def\PYG@tc##1{\textcolor[rgb]{0.13,0.50,0.31}{##1}}}
\expandafter\def\csname PYG@tok@gp\endcsname{\let\PYG@bf=\textbf\def\PYG@tc##1{\textcolor[rgb]{0.78,0.36,0.04}{##1}}}
\expandafter\def\csname PYG@tok@sh\endcsname{\def\PYG@tc##1{\textcolor[rgb]{0.25,0.44,0.63}{##1}}}
\expandafter\def\csname PYG@tok@ow\endcsname{\let\PYG@bf=\textbf\def\PYG@tc##1{\textcolor[rgb]{0.00,0.44,0.13}{##1}}}
\expandafter\def\csname PYG@tok@sx\endcsname{\def\PYG@tc##1{\textcolor[rgb]{0.78,0.36,0.04}{##1}}}
\expandafter\def\csname PYG@tok@bp\endcsname{\def\PYG@tc##1{\textcolor[rgb]{0.00,0.44,0.13}{##1}}}
\expandafter\def\csname PYG@tok@c1\endcsname{\let\PYG@it=\textit\def\PYG@tc##1{\textcolor[rgb]{0.25,0.50,0.56}{##1}}}
\expandafter\def\csname PYG@tok@o\endcsname{\def\PYG@tc##1{\textcolor[rgb]{0.40,0.40,0.40}{##1}}}
\expandafter\def\csname PYG@tok@kc\endcsname{\let\PYG@bf=\textbf\def\PYG@tc##1{\textcolor[rgb]{0.00,0.44,0.13}{##1}}}
\expandafter\def\csname PYG@tok@c\endcsname{\let\PYG@it=\textit\def\PYG@tc##1{\textcolor[rgb]{0.25,0.50,0.56}{##1}}}
\expandafter\def\csname PYG@tok@mf\endcsname{\def\PYG@tc##1{\textcolor[rgb]{0.13,0.50,0.31}{##1}}}
\expandafter\def\csname PYG@tok@err\endcsname{\def\PYG@bc##1{\setlength{\fboxsep}{0pt}\fcolorbox[rgb]{1.00,0.00,0.00}{1,1,1}{\strut ##1}}}
\expandafter\def\csname PYG@tok@mb\endcsname{\def\PYG@tc##1{\textcolor[rgb]{0.13,0.50,0.31}{##1}}}
\expandafter\def\csname PYG@tok@ss\endcsname{\def\PYG@tc##1{\textcolor[rgb]{0.32,0.47,0.09}{##1}}}
\expandafter\def\csname PYG@tok@sr\endcsname{\def\PYG@tc##1{\textcolor[rgb]{0.14,0.33,0.53}{##1}}}
\expandafter\def\csname PYG@tok@mo\endcsname{\def\PYG@tc##1{\textcolor[rgb]{0.13,0.50,0.31}{##1}}}
\expandafter\def\csname PYG@tok@kd\endcsname{\let\PYG@bf=\textbf\def\PYG@tc##1{\textcolor[rgb]{0.00,0.44,0.13}{##1}}}
\expandafter\def\csname PYG@tok@mi\endcsname{\def\PYG@tc##1{\textcolor[rgb]{0.13,0.50,0.31}{##1}}}
\expandafter\def\csname PYG@tok@kn\endcsname{\let\PYG@bf=\textbf\def\PYG@tc##1{\textcolor[rgb]{0.00,0.44,0.13}{##1}}}
\expandafter\def\csname PYG@tok@cpf\endcsname{\let\PYG@it=\textit\def\PYG@tc##1{\textcolor[rgb]{0.25,0.50,0.56}{##1}}}
\expandafter\def\csname PYG@tok@kr\endcsname{\let\PYG@bf=\textbf\def\PYG@tc##1{\textcolor[rgb]{0.00,0.44,0.13}{##1}}}
\expandafter\def\csname PYG@tok@s\endcsname{\def\PYG@tc##1{\textcolor[rgb]{0.25,0.44,0.63}{##1}}}
\expandafter\def\csname PYG@tok@kp\endcsname{\def\PYG@tc##1{\textcolor[rgb]{0.00,0.44,0.13}{##1}}}
\expandafter\def\csname PYG@tok@w\endcsname{\def\PYG@tc##1{\textcolor[rgb]{0.73,0.73,0.73}{##1}}}
\expandafter\def\csname PYG@tok@kt\endcsname{\def\PYG@tc##1{\textcolor[rgb]{0.56,0.13,0.00}{##1}}}
\expandafter\def\csname PYG@tok@sc\endcsname{\def\PYG@tc##1{\textcolor[rgb]{0.25,0.44,0.63}{##1}}}
\expandafter\def\csname PYG@tok@sb\endcsname{\def\PYG@tc##1{\textcolor[rgb]{0.25,0.44,0.63}{##1}}}
\expandafter\def\csname PYG@tok@k\endcsname{\let\PYG@bf=\textbf\def\PYG@tc##1{\textcolor[rgb]{0.00,0.44,0.13}{##1}}}
\expandafter\def\csname PYG@tok@se\endcsname{\let\PYG@bf=\textbf\def\PYG@tc##1{\textcolor[rgb]{0.25,0.44,0.63}{##1}}}
\expandafter\def\csname PYG@tok@sd\endcsname{\let\PYG@it=\textit\def\PYG@tc##1{\textcolor[rgb]{0.25,0.44,0.63}{##1}}}

\def\PYGZbs{\char`\\}
\def\PYGZus{\char`\_}
\def\PYGZob{\char`\{}
\def\PYGZcb{\char`\}}
\def\PYGZca{\char`\^}
\def\PYGZam{\char`\&}
\def\PYGZlt{\char`\<}
\def\PYGZgt{\char`\>}
\def\PYGZsh{\char`\#}
\def\PYGZpc{\char`\%}
\def\PYGZdl{\char`\$}
\def\PYGZhy{\char`\-}
\def\PYGZsq{\char`\'}
\def\PYGZdq{\char`\"}
\def\PYGZti{\char`\~}
% for compatibility with earlier versions
\def\PYGZat{@}
\def\PYGZlb{[}
\def\PYGZrb{]}
\makeatother

\renewcommand\PYGZsq{\textquotesingle}

\begin{document}

\maketitle
\tableofcontents
\phantomsection\label{index::doc}



\chapter{Synopsis}
\label{index:synopsis}\label{index:modular-arbitrary-order-ocean-atmosphere-model-maooam-python-implementation}
This repository provides the code of the model MAOOAM in python. It is a low-order ocean-atmosphere model with an arbitrary expansion of the Fourier modes of temperatures and streamfunctions.
The code in Python is a translation of the Fortran code available in the main Git repository : \href{https://github.com/Climdyn/MAOOAM}{https://github.com/Climdyn/MAOOAM}.


\chapter{Motivation}
\label{index:motivation}
The code has been translated in Python to be used with the Data Assimilation module DAPPER : \href{https://github.com/nansencenter/DAPPER}{https://github.com/nansencenter/DAPPER} .


\chapter{Installation}
\label{index:installation}
The program can be run with python 2.7 or 3.5. Please note that python 3.5 is needed by the Data Assimilation module DAPPER.
Optionally, F2py is also needed to compile the optimized fortran part.
\begin{description}
\item[{To install, unpack the archive in a folder or clone with git:}] \leavevmode
\begin{Verbatim}[commandchars=\\\{\}]
\PYG{g+gp}{\PYGZgt{}\PYGZgt{}\PYGZgt{} }\PYG{n}{git} \PYG{n}{clone} \PYG{n}{https}\PYG{p}{:}\PYG{o}{/}\PYG{o}{/}\PYG{n}{github}\PYG{o}{.}\PYG{n}{com}\PYG{o}{/}\PYG{n}{Climdyn}\PYG{o}{/}\PYG{n}{MAOOAM}\PYG{o}{.}\PYG{n}{git}
\end{Verbatim}

\item[{and run f2py (optional):}] \leavevmode
\begin{Verbatim}[commandchars=\\\{\}]
\PYG{g+gp}{\PYGZgt{}\PYGZgt{}\PYGZgt{} }\PYG{n}{f2py} \PYG{o}{\PYGZhy{}}\PYG{n}{c} \PYG{n}{sparse\PYGZus{}mult}\PYG{o}{.}\PYG{n}{pyf} \PYG{n}{sparse\PYGZus{}mult}\PYG{o}{.}\PYG{n}{f90}
\end{Verbatim}

\end{description}


\chapter{Getting started}
\label{index:getting-started}
The user first has to fill the params\_maooam.py according to their needs. See its documentation for more information about this file.
Some examples related to already published articles are available in the params folder.

Finally, the ic.py file specifying the initial condition should be defined. To
obtain an example of this configuration file corresponding to the model you
have previously defined, simply delete the current ic.py file (if it exists)
and run the program:

\begin{Verbatim}[commandchars=\\\{\}]
\PYG{g+gp}{\PYGZgt{}\PYGZgt{}\PYGZgt{} }\PYG{n}{ipython} \PYG{n}{maooam}\PYG{o}{.}\PYG{n}{py}
\end{Verbatim}

It will generate a new one and start with the 0 initial condition. If you want another
initial condition, stop the program, fill the newly generated file and restart:

\begin{Verbatim}[commandchars=\\\{\}]
\PYG{g+gp}{\PYGZgt{}\PYGZgt{}\PYGZgt{} }\PYG{n}{ipython} \PYG{n}{maooam}\PYG{o}{.}\PYG{n}{py}
\end{Verbatim}

The code will generate a file evol\_field.dat containing the recorded time evolution of the variables.


\chapter{Description of the files}
\label{index:description-of-the-files}\begin{itemize}
\item {} 
maooam.py : main program.

\item {} 
params\_maooam.py : a module for the parameters of the model (dimensional, integral and physical).

\item {} 
ic.py : initial conditions file.

\item {} 
ic\_def.py a module that generate the initial conditions if it does not exist.

\item {} 
inprod\_analytic.py : a module that compute the inner product needed for the tensor computation.

\item {} 
aotensor.py : a module that compute the tensor.

\item {} 
integrator.py: a module that compute one step of the model and RK2 integration.

\item {} 
sparse\_mult.f90 and sparse\_mult.pyf : fortran and f2py files to call fortran module in python.

\end{itemize}


\chapter{Contents}
\label{index:contents}\phantomsection\label{rstfiles/params_maooam:module-params_maooam}\index{params\_maooam (module)}

\section{Parameters module}
\label{rstfiles/params_maooam:parameters-module}\label{rstfiles/params_maooam::doc}
This module defines the parameters for the model.

\begin{notice}{note}{Note:}
The python code is available here :     params\_maooam.py .
\end{notice}
\begin{quote}\begin{description}
\item[{Example}] \leavevmode
\end{description}\end{quote}

\begin{Verbatim}[commandchars=\\\{\}]
\PYG{g+gp}{\PYGZgt{}\PYGZgt{}\PYGZgt{} }\PYG{k+kn}{from} \PYG{n+nn}{params\PYGZus{}maooam} \PYG{k+kn}{import} \PYG{n}{ndim}\PYG{p}{,}\PYG{n}{natm}\PYG{p}{,}\PYG{n}{noc}
\PYG{g+gp}{\PYGZgt{}\PYGZgt{}\PYGZgt{} }\PYG{k+kn}{from} \PYG{n+nn}{params\PYGZus{}maooam} \PYG{k+kn}{import} \PYG{n}{oms}\PYG{p}{,}\PYG{n}{nboc}\PYG{p}{,}\PYG{n}{ams}\PYG{p}{,}\PYG{n}{nbatm}
\PYG{g+gp}{\PYGZgt{}\PYGZgt{}\PYGZgt{} }\PYG{k+kn}{from} \PYG{n+nn}{params\PYGZus{}maooam} \PYG{k+kn}{import} \PYG{o}{*}
\end{Verbatim}

There are three types of parameters :
\begin{itemize}
\item {} 
integration parameters : simulation time (transient and effective),     time step, writeout and write step time

\item {} 
dimensional parameters : dimensions of the truncation of fourier     for the atmosphere and the ocean

\item {} 
physical parameters : they are used in the tensor for the integration

\end{itemize}


\subsection{Integration parameters}
\label{rstfiles/params_maooam:integration-parameters}
\begin{notice}{warning}{Warning:}
Time is adimensional. If t\_real is in seconds,
then t\_model = t\_real * f\_0 where f\_0 is the Coriolis
parameter at 45 degrees latitude ( 1.032e-4 )
\end{notice}
\begin{itemize}
\item {} 
\textbf{t\_trans} : the transient simulation time of the model to be on the attractor. The states vectors are not written on evol\_field.dat.

\item {} 
\textbf{t\_run} : the running simulation time of the model. The states vectors are written on evol\_field.dat every tw.

\item {} 
\textbf{dt} : the step time.

\item {} 
\textbf{writeout} : boolean value to decide if the module produces evol\_field.dat.

\item {} 
\textbf{tw} : the step time to write on evol\_field.

\item {} 
\textbf{f2py} : boolean to activate the f2py optimization.

\end{itemize}


\subsection{Dimensional parameters}
\label{rstfiles/params_maooam:dimensional-parameters}\begin{itemize}
\item {} 
\textbf{oms} and \textbf{ams} : the matrices that gives the possible values of the modes Nx and Ny.

\item {} 
\textbf{nboc} and \textbf{natm} : the numbers of oceanic and atmospheric blocs.

\item {} 
\textbf{natm} and \textbf{noc} : the numbers of functions available.

\item {} 
\textbf{ndim} : the total dimension.

\end{itemize}
\begin{quote}\begin{description}
\item[{Example}] \leavevmode
\end{description}\end{quote}

\begin{Verbatim}[commandchars=\\\{\}]
\PYG{g+gp}{\PYGZgt{}\PYGZgt{}\PYGZgt{} }\PYG{n}{oms} \PYG{o}{=}\PYG{n}{get\PYGZus{}modes}\PYG{p}{(}\PYG{l+m+mi}{2}\PYG{p}{,}\PYG{l+m+mi}{4}\PYG{p}{)}\PYG{c+c1}{\PYGZsh{} ocean mode selection}
\PYG{g+gp}{\PYGZgt{}\PYGZgt{}\PYGZgt{} }\PYG{n}{ams} \PYG{o}{=}\PYG{n}{get\PYGZus{}modes}\PYG{p}{(}\PYG{l+m+mi}{2}\PYG{p}{,}\PYG{l+m+mi}{2}\PYG{p}{)}\PYG{c+c1}{\PYGZsh{} atmosphere mode selection}
\PYG{g+gp}{\PYGZgt{}\PYGZgt{}\PYGZgt{} }\PYG{n}{nboc}\PYG{p}{,}\PYG{n}{nbatm} \PYG{o}{=} \PYG{l+m+mi}{2}\PYG{o}{*}\PYG{l+m+mi}{4}\PYG{p}{,}\PYG{l+m+mi}{2}\PYG{o}{*}\PYG{l+m+mi}{2}      \PYG{c+c1}{\PYGZsh{} number of blocks}
\PYG{g+gp}{\PYGZgt{}\PYGZgt{}\PYGZgt{} }\PYG{p}{(}\PYG{n}{natm}\PYG{p}{,}\PYG{n}{noc}\PYG{p}{,}\PYG{n}{ndim}\PYG{p}{)}\PYG{o}{=}\PYG{n}{init\PYGZus{}params}\PYG{p}{(}\PYG{n}{nboc}\PYG{p}{,}\PYG{n}{nbatm}\PYG{p}{)}
\PYG{g+go}{\PYGZgt{}\PYGZgt{}\PYGZgt{}}
\PYG{g+gp}{\PYGZgt{}\PYGZgt{}\PYGZgt{} }\PYG{c+c1}{\PYGZsh{} Oceanic blocs}
\PYG{g+gp}{\PYGZgt{}\PYGZgt{}\PYGZgt{} }\PYG{c+c1}{\PYGZsh{}( x block number accounts for half\PYGZhy{}integer wavenumber e.g 1    =\PYGZgt{} 1/2 , 2 =\PYGZgt{} 1, etc...)}
\PYG{g+gp}{\PYGZgt{}\PYGZgt{}\PYGZgt{} }\PYG{n}{OMS}\PYG{p}{[}\PYG{l+m+mi}{0}\PYG{p}{,}\PYG{p}{:}\PYG{p}{]} \PYG{o}{=} \PYG{l+m+mi}{1}\PYG{p}{,}\PYG{l+m+mi}{1}
\PYG{g+gp}{\PYGZgt{}\PYGZgt{}\PYGZgt{} }\PYG{n}{OMS}\PYG{p}{[}\PYG{l+m+mi}{1}\PYG{p}{,}\PYG{p}{:}\PYG{p}{]} \PYG{o}{=} \PYG{l+m+mi}{1}\PYG{p}{,}\PYG{l+m+mi}{2}
\PYG{g+gp}{\PYGZgt{}\PYGZgt{}\PYGZgt{} }\PYG{n}{OMS}\PYG{p}{[}\PYG{l+m+mi}{2}\PYG{p}{,}\PYG{p}{:}\PYG{p}{]} \PYG{o}{=} \PYG{l+m+mi}{1}\PYG{p}{,}\PYG{l+m+mi}{3}
\PYG{g+gp}{\PYGZgt{}\PYGZgt{}\PYGZgt{} }\PYG{n}{OMS}\PYG{p}{[}\PYG{l+m+mi}{3}\PYG{p}{,}\PYG{p}{:}\PYG{p}{]} \PYG{o}{=} \PYG{l+m+mi}{1}\PYG{p}{,}\PYG{l+m+mi}{4}
\PYG{g+gp}{\PYGZgt{}\PYGZgt{}\PYGZgt{} }\PYG{n}{OMS}\PYG{p}{[}\PYG{l+m+mi}{4}\PYG{p}{,}\PYG{p}{:}\PYG{p}{]} \PYG{o}{=} \PYG{l+m+mi}{2}\PYG{p}{,}\PYG{l+m+mi}{1}
\PYG{g+gp}{\PYGZgt{}\PYGZgt{}\PYGZgt{} }\PYG{n}{OMS}\PYG{p}{[}\PYG{l+m+mi}{5}\PYG{p}{,}\PYG{p}{:}\PYG{p}{]} \PYG{o}{=} \PYG{l+m+mi}{2}\PYG{p}{,}\PYG{l+m+mi}{2}
\PYG{g+gp}{\PYGZgt{}\PYGZgt{}\PYGZgt{} }\PYG{n}{OMS}\PYG{p}{[}\PYG{l+m+mi}{6}\PYG{p}{,}\PYG{p}{:}\PYG{p}{]} \PYG{o}{=} \PYG{l+m+mi}{2}\PYG{p}{,}\PYG{l+m+mi}{3}
\PYG{g+gp}{\PYGZgt{}\PYGZgt{}\PYGZgt{} }\PYG{n}{OMS}\PYG{p}{[}\PYG{l+m+mi}{7}\PYG{p}{,}\PYG{p}{:}\PYG{p}{]} \PYG{o}{=} \PYG{l+m+mi}{2}\PYG{p}{,}\PYG{l+m+mi}{4}
\PYG{g+gp}{\PYGZgt{}\PYGZgt{}\PYGZgt{} }\PYG{c+c1}{\PYGZsh{}Atmospheric blocs}
\PYG{g+gp}{\PYGZgt{}\PYGZgt{}\PYGZgt{} }\PYG{n}{AMS}\PYG{p}{[}\PYG{l+m+mi}{0}\PYG{p}{,}\PYG{p}{:}\PYG{p}{]} \PYG{o}{=} \PYG{l+m+mi}{1}\PYG{p}{,}\PYG{l+m+mi}{1}
\PYG{g+gp}{\PYGZgt{}\PYGZgt{}\PYGZgt{} }\PYG{n}{AMS}\PYG{p}{[}\PYG{l+m+mi}{1}\PYG{p}{,}\PYG{p}{:}\PYG{p}{]} \PYG{o}{=} \PYG{l+m+mi}{1}\PYG{p}{,}\PYG{l+m+mi}{2}
\PYG{g+gp}{\PYGZgt{}\PYGZgt{}\PYGZgt{} }\PYG{n}{AMS}\PYG{p}{[}\PYG{l+m+mi}{2}\PYG{p}{,}\PYG{p}{:}\PYG{p}{]} \PYG{o}{=} \PYG{l+m+mi}{2}\PYG{p}{,}\PYG{l+m+mi}{1}
\PYG{g+gp}{\PYGZgt{}\PYGZgt{}\PYGZgt{} }\PYG{n}{AMS}\PYG{p}{[}\PYG{l+m+mi}{3}\PYG{p}{,}\PYG{p}{:}\PYG{p}{]} \PYG{o}{=} \PYG{l+m+mi}{2}\PYG{p}{,}\PYG{l+m+mi}{2}
\end{Verbatim}
\begin{quote}\begin{description}
\item[{Typical dimensional parameters}] \leavevmode
\end{description}\end{quote}
\begin{itemize}
\item {} 
atmosphere 2x,2y ; ocean 2x,4y ; ndim = 36

\item {} 
atmosphere 2x,2y ; ocean 4x,4y ; ndim = 52

\item {} 
atmosphere 2x,4y ; ocean 2x,4y ; ndim = 56

\end{itemize}


\subsection{Physical parameters}
\label{rstfiles/params_maooam:physical-parameters}
Some defaut parameters are presented below.
Some parameters files related to already published article are available in the params folder.


\subsubsection{Scale parameters}
\label{rstfiles/params_maooam:scale-parameters}\begin{itemize}
\item {} 
\textbf{scale = 5.e6}  : characteristic space scale, L*pi

\item {} 
\textbf{f0 = 1.032e-4}  : Coriolis parameter at 45 degrees latitude

\item {} 
\textbf{n = 1.5e0}  : aspect ratio (n = 2Ly/Lx ; Lx = 2*pi*L/n; Ly = pi*L)

\item {} 
\textbf{rra = 6370.e3}  : earth radius

\item {} 
\textbf{phi0\_npi = 0.25e0}  : latitude exprimed in fraction of pi

\end{itemize}


\subsubsection{Parameters for the ocean}
\label{rstfiles/params_maooam:parameters-for-the-ocean}\begin{itemize}
\item {} 
\textbf{gp = 3.1e-2}  : reduced gravity

\item {} 
\textbf{r = 1.e-8}  : frictional coefficient at the bottom of the ocean

\item {} 
\textbf{h = 5.e2}  : depth of the water layer of the ocean

\item {} 
\textbf{d = 1.e-8}  : the coupling parameter (should be divided by f0 to be adim)

\end{itemize}


\subsubsection{Parameters for the atmosphere}
\label{rstfiles/params_maooam:parameters-for-the-atmosphere}\begin{itemize}
\item {} 
\textbf{k = 0.02}  : atmosphere bottom friction coefficient

\item {} 
\textbf{kp = 0.04}  : atmosphere internal friction coefficient

\item {} 
\textbf{sig0 = 0.1e0}  : static stability of the atmosphere

\end{itemize}


\subsubsection{Temperature-related parameters for the ocean}
\label{rstfiles/params_maooam:temperature-related-parameters-for-the-ocean}\begin{itemize}
\item {} 
\textbf{Go = 2.e8}  : Specific heat capacity of the ocean (50m layer)

\item {} 
\textbf{Co = 350}  : Constant short-wave radiation of the ocean

\item {} 
\textbf{To0 = 285.0}  : Stationary solution for the 0-th order ocean temperature

\end{itemize}


\subsubsection{Temperature-related parameters for the atmosphere}
\label{rstfiles/params_maooam:temperature-related-parameters-for-the-atmosphere}\begin{itemize}
\item {} 
\textbf{Ga = 1.e7}  : Specific heat capacity of the atmosphere

\item {} 
\textbf{Ca = 100.e0}  ; Constant short-wave radiation of the atmosphere

\item {} 
\textbf{epsa = 0.76e0}  : Emissivity coefficient for the grey-body atmosphere

\item {} 
\textbf{Ta0 = 270.0}  : Stationary solution for the 0-th order atmospheric temperature

\end{itemize}


\subsubsection{Other temperature-related parameters/constants}
\label{rstfiles/params_maooam:other-temperature-related-parameters-constants}\begin{itemize}
\item {} 
\textbf{sc = 1.}  : Ratio of surface to atmosphere temperature

\item {} 
\textbf{lambda = 20.00}  : Sensible+turbulent heat exchange between oc and atm

\item {} 
\textbf{rr = 287.e0}  : Gas constant of dry air

\item {} 
\textbf{sb = 5.6e-8}  : Stefan-Boltzmann constant

\end{itemize}


\subsubsection{Key values}
\label{rstfiles/params_maooam:key-values}\begin{itemize}
\item {} 
\textbf{k} is the friction coefficient at the bottom of the atmosphere. Typical values are 0.01 or 0.0145 for chaotic regimes.

\item {} 
\textbf{kp} is the internal friction between the atmosphere layers. kp=2*k

\item {} 
\textbf{d} is the friction coefficient between the ocean and the atmosphere. Typical values are 6*10\textasciicircum{}\{-8\} s\textasciicircum{}\{-1\} or 9*10\textasciicircum{}\{-8\} s\textasciicircum{}\{-1\}.

\item {} 
\textbf{lambda} is the heat exchange between the ocean and the atmosphere. Typical values are 10 W m\textasciicircum{}\{-2\} K\textasciicircum{}\{-1\} or 15.06 W m \textasciicircum{}\{-2\} K\textasciicircum{}\{-1\}.

\end{itemize}


\subsection{Dependencies}
\label{rstfiles/params_maooam:dependencies}
\begin{Verbatim}[commandchars=\\\{\}]
\PYG{g+gp}{\PYGZgt{}\PYGZgt{}\PYGZgt{} }\PYG{k+kn}{import} \PYG{n+nn}{numpy} \PYG{k+kn}{as} \PYG{n+nn}{np}
\end{Verbatim}


\subsection{Fonctions}
\label{rstfiles/params_maooam:fonctions}
Here are the functions to generate the parameters.
\index{get\_modes() (in module params\_maooam)}

\begin{fulllineitems}
\phantomsection\label{rstfiles/params_maooam:params_maooam.get_modes}\pysiglinewithargsret{\code{params\_maooam.}\bfcode{get\_modes}}{\emph{nxmax}, \emph{nymax}}{}
Computes the matrix oms and ams with nxmax and nymax

\end{fulllineitems}

\index{init\_params() (in module params\_maooam)}

\begin{fulllineitems}
\phantomsection\label{rstfiles/params_maooam:params_maooam.init_params}\pysiglinewithargsret{\code{params\_maooam.}\bfcode{init\_params}}{\emph{nboc}, \emph{nbatm}}{}
Computes the dimensions of the system

\end{fulllineitems}

\phantomsection\label{rstfiles/ic_def:module-ic_def}\index{ic\_def (module)}

\section{Initial conditions generator module}
\label{rstfiles/ic_def:initial-conditions-generator-module}\label{rstfiles/ic_def::doc}
This module generates the initial conditions for the model if it doesn't     exist with the good dimensions.

The dimensions of the system can be changed in the parameters file params\_maooam.py .
Then delete ic.py and ic\_def.py will regenerates it.

\begin{notice}{note}{Note:}
The python code is available here :    ic\_def.py .
\end{notice}
\begin{quote}\begin{description}
\item[{Example}] \leavevmode
\end{description}\end{quote}

\begin{Verbatim}[commandchars=\\\{\}]
\PYG{g+gp}{\PYGZgt{}\PYGZgt{}\PYGZgt{} }\PYG{k+kn}{from} \PYG{n+nn}{ic\PYGZus{}def} \PYG{k+kn}{import} \PYG{n}{load\PYGZus{}IC}
\PYG{g+gp}{\PYGZgt{}\PYGZgt{}\PYGZgt{} }\PYG{n}{load\PYGZus{}IC}\PYG{p}{(}\PYG{p}{)}
\end{Verbatim}


\subsection{Global file}
\label{rstfiles/ic_def:global-file}
The file ic.py in the same directory.


\subsection{Dependencies}
\label{rstfiles/ic_def:dependencies}
Uses the following modules to know the dimensions :

\begin{Verbatim}[commandchars=\\\{\}]
\PYG{g+gp}{\PYGZgt{}\PYGZgt{}\PYGZgt{} }\PYG{k+kn}{from} \PYG{n+nn}{params\PYGZus{}maooam} \PYG{k+kn}{import} \PYG{n}{natm}\PYG{p}{,} \PYG{n}{noc}\PYG{p}{,} \PYG{n}{ndim}\PYG{p}{,} \PYG{n}{t\PYGZus{}trans}\PYG{p}{,} \PYG{n}{t\PYGZus{}run}
\PYG{g+gp}{\PYGZgt{}\PYGZgt{}\PYGZgt{} }\PYG{k+kn}{from} \PYG{n+nn}{inprod\PYGZus{}analytic} \PYG{k+kn}{import} \PYG{n}{awavenum}\PYG{p}{,} \PYG{n}{owavenum}\PYG{p}{,} \PYG{n}{init\PYGZus{}inprod}
\end{Verbatim}


\subsection{Functions}
\label{rstfiles/ic_def:functions}
Functions in the module :
\index{load\_IC() (in module ic\_def)}

\begin{fulllineitems}
\phantomsection\label{rstfiles/ic_def:ic_def.load_IC}\pysiglinewithargsret{\code{ic\_def.}\bfcode{load\_IC}}{}{}
Check if ic.py exists, if not creates it with good dimensions and
zero initial conditions

\end{fulllineitems}

\phantomsection\label{rstfiles/ic:module-ic}\index{ic (module)}

\section{Initial conditions file}
\label{rstfiles/ic::doc}\label{rstfiles/ic:initial-conditions-file}
This file defines the initial conditions of the model.         To be deleted if the dimensions are changed.
\begin{quote}\begin{description}
\item[{Example}] \leavevmode
\end{description}\end{quote}

\begin{Verbatim}[commandchars=\\\{\}]
\PYG{g+gp}{\PYGZgt{}\PYGZgt{}\PYGZgt{} }\PYG{k+kn}{from} \PYG{n+nn}{ic} \PYG{k+kn}{import} \PYG{n}{X0}
\end{Verbatim}


\subsection{Global variables (state vectors)}
\label{rstfiles/ic:global-variables-state-vectors}\begin{itemize}
\item {} 
X0 random (non-null) initial conditions.

\end{itemize}


\subsection{Dependencies}
\label{rstfiles/ic:dependencies}
\begin{Verbatim}[commandchars=\\\{\}]
\PYG{g+gp}{\PYGZgt{}\PYGZgt{}\PYGZgt{} }\PYG{k+kn}{import} \PYG{n+nn}{numpy} \PYG{k+kn}{as} \PYG{n+nn}{np}
\end{Verbatim}
\phantomsection\label{rstfiles/inprod_analytic:module-inprod_analytic}\index{inprod\_analytic (module)}

\section{Inner products module}
\label{rstfiles/inprod_analytic:inner-products-module}\label{rstfiles/inprod_analytic::doc}
Inner products between the truncated set of basis functions for the 
ocean and atmosphere streamfunction fields.

\begin{notice}{note}{Note:}
These are calculated using the analytical expressions from       De Cruz, L., Demaeyer, J. and Vannitsem, S.: A modular         arbitrary-order ocean-atmosphere model: MAOOAM v1.0, Geosci. 
Model Dev. Discuss.     and from        Cehelsky, P., \& Tung, K. K. : Theories of multiple equilibria and       weather regimes-A critical reexamination. Part II: Baroclinic two-layer         models. Journal of the atmospheric sciences, 44(21), 3282-3303, 1987.
\end{notice}

\begin{notice}{note}{Note:}
The python code is available here :          inprod\_analytic.py .
\end{notice}
\begin{quote}\begin{description}
\item[{Example}] \leavevmode
\end{description}\end{quote}

\begin{Verbatim}[commandchars=\\\{\}]
\PYG{g+gp}{\PYGZgt{}\PYGZgt{}\PYGZgt{} }\PYG{k+kn}{from} \PYG{n+nn}{inprod\PYGZus{}analytic} \PYG{k+kn}{import} \PYG{n}{init\PYGZus{}inprod}
\PYG{g+gp}{\PYGZgt{}\PYGZgt{}\PYGZgt{} }\PYG{n}{init\PYGZus{}inprod}\PYG{p}{(}\PYG{p}{)}
\end{Verbatim}


\subsection{Global variables}
\label{rstfiles/inprod_analytic:global-variables}
\begin{Verbatim}[commandchars=\\\{\}]
\PYG{g+gp}{\PYGZgt{}\PYGZgt{}\PYGZgt{} }\PYG{n}{awavenum} \PYG{o}{=} \PYG{n}{np}\PYG{o}{.}\PYG{n}{empty}\PYG{p}{(}\PYG{n}{natm}\PYG{p}{,} \PYG{n}{dtype}\PYG{o}{=}\PYG{n+nb}{object}\PYG{p}{)}
\PYG{g+gp}{\PYGZgt{}\PYGZgt{}\PYGZgt{} }\PYG{n}{owavenum} \PYG{o}{=} \PYG{n}{np}\PYG{o}{.}\PYG{n}{empty}\PYG{p}{(}\PYG{n}{noc}\PYG{p}{,} \PYG{n}{dtype}\PYG{o}{=}\PYG{n+nb}{object}\PYG{p}{)}
\PYG{g+gp}{\PYGZgt{}\PYGZgt{}\PYGZgt{} }\PYG{n}{atmos} \PYG{o}{=} \PYG{n}{atm\PYGZus{}tensors}\PYG{p}{(}\PYG{n}{natm}\PYG{p}{)}
\PYG{g+gp}{\PYGZgt{}\PYGZgt{}\PYGZgt{} }\PYG{n}{ocean} \PYG{o}{=} \PYG{n}{ocean\PYGZus{}tensors}\PYG{p}{(}\PYG{n}{noc}\PYG{p}{)}
\end{Verbatim}


\subsection{Dependencies}
\label{rstfiles/inprod_analytic:dependencies}
it uses the modules :

\begin{Verbatim}[commandchars=\\\{\}]
\PYG{g+gp}{\PYGZgt{}\PYGZgt{}\PYGZgt{} }\PYG{k+kn}{import} \PYG{n+nn}{numpy} \PYG{k+kn}{as} \PYG{n+nn}{np}
\PYG{g+gp}{\PYGZgt{}\PYGZgt{}\PYGZgt{} }\PYG{k+kn}{from} \PYG{n+nn}{scipy.sparse} \PYG{k+kn}{import} \PYG{n}{csr\PYGZus{}matrix}
\end{Verbatim}

\begin{Verbatim}[commandchars=\\\{\}]
\PYG{g+gp}{\PYGZgt{}\PYGZgt{}\PYGZgt{} }\PYG{k+kn}{from} \PYG{n+nn}{params\PYGZus{}maooam} \PYG{k+kn}{import} \PYG{n}{nbatm}
\PYG{g+gp}{\PYGZgt{}\PYGZgt{}\PYGZgt{} }\PYG{k+kn}{from} \PYG{n+nn}{params\PYGZus{}maooam} \PYG{k+kn}{import} \PYG{n}{nboc}
\PYG{g+gp}{\PYGZgt{}\PYGZgt{}\PYGZgt{} }\PYG{k+kn}{from} \PYG{n+nn}{params\PYGZus{}maooam} \PYG{k+kn}{import} \PYG{n}{natm}
\PYG{g+gp}{\PYGZgt{}\PYGZgt{}\PYGZgt{} }\PYG{k+kn}{from} \PYG{n+nn}{params\PYGZus{}maooam} \PYG{k+kn}{import} \PYG{n}{noc}
\PYG{g+gp}{\PYGZgt{}\PYGZgt{}\PYGZgt{} }\PYG{k+kn}{from} \PYG{n+nn}{params\PYGZus{}maooam} \PYG{k+kn}{import} \PYG{n}{n}
\PYG{g+gp}{\PYGZgt{}\PYGZgt{}\PYGZgt{} }\PYG{k+kn}{from} \PYG{n+nn}{params\PYGZus{}maooam} \PYG{k+kn}{import} \PYG{n}{oms}
\PYG{g+gp}{\PYGZgt{}\PYGZgt{}\PYGZgt{} }\PYG{k+kn}{from} \PYG{n+nn}{params\PYGZus{}maooam} \PYG{k+kn}{import} \PYG{n}{ams}
\PYG{g+gp}{\PYGZgt{}\PYGZgt{}\PYGZgt{} }\PYG{k+kn}{from} \PYG{n+nn}{params\PYGZus{}maooam} \PYG{k+kn}{import} \PYG{n}{pi}
\end{Verbatim}


\subsection{Classes}
\label{rstfiles/inprod_analytic:classes}\begin{itemize}
\item {} 
atm\_wavenum(typ,P,N,H,Nx,Ny)

\item {} 
ocean\_wavenum(P,H,Nx,Ny)

\item {} 
atm\_tensors(natm)

\item {} 
ocean\_tensors(noc)

\end{itemize}
\index{atm\_tensors (class in inprod\_analytic)}

\begin{fulllineitems}
\phantomsection\label{rstfiles/inprod_analytic:inprod_analytic.atm_tensors}\pysiglinewithargsret{\strong{class }\code{inprod\_analytic.}\bfcode{atm\_tensors}}{\emph{natm}}{}
Class which contains all the coefficients
a,c,d,s,b,g needed for the tensor computation :

Attributes :
\begin{itemize}
\item {} 
\(a_{i, j}\)

\item {} 
\(c_{i, j}\)

\item {} 
\(d_{i, j}\)

\item {} 
\(s_{i, j}\)

\item {} 
\(b_{i, j, k}\)

\item {} 
\(g_{i, j, k}\)

\end{itemize}

Return :
\begin{itemize}
\item {} 
The object will be name \emph{atmos}.

\end{itemize}
\index{calculate\_a() (inprod\_analytic.atm\_tensors method)}

\begin{fulllineitems}
\phantomsection\label{rstfiles/inprod_analytic:inprod_analytic.atm_tensors.calculate_a}\pysiglinewithargsret{\bfcode{calculate\_a}}{}{}~\begin{gather}
\begin{split}a_{i, j} = (F_i, {\nabla}^2 F_j)\end{split}\notag
\end{gather}
\begin{notice}{note}{Note:}
Eigenvalues of the Laplacian (atmospheric)
\end{notice}

\end{fulllineitems}

\index{calculate\_b() (inprod\_analytic.atm\_tensors method)}

\begin{fulllineitems}
\phantomsection\label{rstfiles/inprod_analytic:inprod_analytic.atm_tensors.calculate_b}\pysiglinewithargsret{\bfcode{calculate\_b}}{}{}~\begin{gather}
\begin{split}b_{i, j, k} = (F_i, J(F_j, \nabla^2 F_k))\end{split}\notag
\end{gather}
\begin{notice}{note}{Note:}
Atmospheric g and a tensors must be computed before calling this routine.
\end{notice}

\end{fulllineitems}

\index{calculate\_c\_atm() (inprod\_analytic.atm\_tensors method)}

\begin{fulllineitems}
\phantomsection\label{rstfiles/inprod_analytic:inprod_analytic.atm_tensors.calculate_c_atm}\pysiglinewithargsret{\bfcode{calculate\_c\_atm}}{}{}~\begin{gather}
\begin{split}c_{i,j} = (F_i, \partial_x F_j)\end{split}\notag
\end{gather}
\begin{notice}{note}{Note:}
Beta term for the atmosphere
Strict function !! Only accepts KL type.
For any other combination, it will not calculate anything.
\end{notice}

\end{fulllineitems}

\index{calculate\_d() (inprod\_analytic.atm\_tensors method)}

\begin{fulllineitems}
\phantomsection\label{rstfiles/inprod_analytic:inprod_analytic.atm_tensors.calculate_d}\pysiglinewithargsret{\bfcode{calculate\_d}}{\emph{ocean}}{}~\begin{gather}
\begin{split}d_{i,j} = (F_i, \nabla^2 \eta_j)\end{split}\notag
\end{gather}
\begin{notice}{note}{Note:}
Forcing of the ocean on the atmosphere.
Atmospheric s tensor and oceanic M tensor must be computed
before calling this routine !
\end{notice}

\end{fulllineitems}

\index{calculate\_g() (inprod\_analytic.atm\_tensors method)}

\begin{fulllineitems}
\phantomsection\label{rstfiles/inprod_analytic:inprod_analytic.atm_tensors.calculate_g}\pysiglinewithargsret{\bfcode{calculate\_g}}{}{}~\begin{gather}
\begin{split}g_{i,j,k} = (F_i, J(F_j, F_k))\end{split}\notag
\end{gather}
\begin{notice}{note}{Note:}
This is a strict function: it only accepts AKL, KKL
and LLL types.
For any other combination, it will not calculate anything.
\end{notice}

\end{fulllineitems}

\index{calculate\_s() (inprod\_analytic.atm\_tensors method)}

\begin{fulllineitems}
\phantomsection\label{rstfiles/inprod_analytic:inprod_analytic.atm_tensors.calculate_s}\pysiglinewithargsret{\bfcode{calculate\_s}}{}{}~\begin{gather}
\begin{split}s_{i,j} = (F_i, \eta_j)\end{split}\notag
\end{gather}
\begin{notice}{note}{Note:}
Forcing (thermal) of the ocean on the atmosphere.
\end{notice}

\end{fulllineitems}


\end{fulllineitems}

\index{atm\_wavenum (class in inprod\_analytic)}

\begin{fulllineitems}
\phantomsection\label{rstfiles/inprod_analytic:inprod_analytic.atm_wavenum}\pysiglinewithargsret{\strong{class }\code{inprod\_analytic.}\bfcode{atm\_wavenum}}{\emph{typ}, \emph{P}, \emph{M}, \emph{H}, \emph{Nx}, \emph{Ny}}{}
Class to define atmosphere wavenumbers.

Attributes :
\begin{itemize}
\item {} 
typ (char) = `A','K' or `L'.

\item {} 
M (int)

\item {} 
P (int)

\item {} 
H (int)

\item {} 
Nx (int)

\item {} 
Ny (int)

\end{itemize}

\end{fulllineitems}

\index{init\_inprod() (in module inprod\_analytic)}

\begin{fulllineitems}
\phantomsection\label{rstfiles/inprod_analytic:inprod_analytic.init_inprod}\pysiglinewithargsret{\code{inprod\_analytic.}\bfcode{init\_inprod}}{}{}
creates and computes the inner products.

\end{fulllineitems}

\index{ocean\_tensors (class in inprod\_analytic)}

\begin{fulllineitems}
\phantomsection\label{rstfiles/inprod_analytic:inprod_analytic.ocean_tensors}\pysiglinewithargsret{\strong{class }\code{inprod\_analytic.}\bfcode{ocean\_tensors}}{\emph{noc}}{}
Class which contains all the coefficients k,m,n,w,o,c     needed for the tensor computation :

Attributes :
\begin{itemize}
\item {} 
\(K_{i,j}\)

\item {} 
\(M_{i,j}\)

\item {} 
\(N_{i,j}\)

\item {} 
\(W_{i,j}\)

\item {} 
\(O_{i,j,k}\)

\item {} 
\(C_{i,j,k}\)

\end{itemize}

Return :
\begin{itemize}
\item {} 
The object will be name ocean

\end{itemize}
\index{calculate\_C\_oc() (inprod\_analytic.ocean\_tensors method)}

\begin{fulllineitems}
\phantomsection\label{rstfiles/inprod_analytic:inprod_analytic.ocean_tensors.calculate_C_oc}\pysiglinewithargsret{\bfcode{calculate\_C\_oc}}{}{}~\begin{gather}
\begin{split}C_{i,j,k} = (\eta_i, J(\eta_j,\nabla^2 \eta_k))\end{split}\notag
\end{gather}
\begin{notice}{note}{Note:}
Requires \(O_{i,j,k}\) 
\end{notice}

and \(M_{i,j}\) to be calculated beforehand.

\end{fulllineitems}

\index{calculate\_K() (inprod\_analytic.ocean\_tensors method)}

\begin{fulllineitems}
\phantomsection\label{rstfiles/inprod_analytic:inprod_analytic.ocean_tensors.calculate_K}\pysiglinewithargsret{\bfcode{calculate\_K}}{\emph{atmos}}{}
Forcing of the atmosphere on the ocean.
\begin{gather}
\begin{split}K_{i,j} = (\eta_i, \nabla^2 F_j)\end{split}\notag
\end{gather}
\begin{notice}{note}{Note:}
atmospheric a and s tensors must be computed before calling
this function !
\end{notice}

\end{fulllineitems}

\index{calculate\_M() (inprod\_analytic.ocean\_tensors method)}

\begin{fulllineitems}
\phantomsection\label{rstfiles/inprod_analytic:inprod_analytic.ocean_tensors.calculate_M}\pysiglinewithargsret{\bfcode{calculate\_M}}{}{}
Forcing of the ocean fields on the ocean.
\begin{gather}
\begin{split}M_{i,j} = (\eta_i, \nabla^2 \eta_j)\end{split}\notag
\end{gather}
\end{fulllineitems}

\index{calculate\_N() (inprod\_analytic.ocean\_tensors method)}

\begin{fulllineitems}
\phantomsection\label{rstfiles/inprod_analytic:inprod_analytic.ocean_tensors.calculate_N}\pysiglinewithargsret{\bfcode{calculate\_N}}{}{}
Beta term for the ocean
\begin{gather}
\begin{split}N_{i,j} = (\eta_i, \partial_x \eta_j)\end{split}\notag
\end{gather}
\end{fulllineitems}

\index{calculate\_O() (inprod\_analytic.ocean\_tensors method)}

\begin{fulllineitems}
\phantomsection\label{rstfiles/inprod_analytic:inprod_analytic.ocean_tensors.calculate_O}\pysiglinewithargsret{\bfcode{calculate\_O}}{}{}
Temperature advection term (passive scalar)
\begin{gather}
\begin{split}O_{i,j,k} = (\eta_i, J(\eta_j, \eta_k))\end{split}\notag
\end{gather}
\end{fulllineitems}

\index{calculate\_W() (inprod\_analytic.ocean\_tensors method)}

\begin{fulllineitems}
\phantomsection\label{rstfiles/inprod_analytic:inprod_analytic.ocean_tensors.calculate_W}\pysiglinewithargsret{\bfcode{calculate\_W}}{\emph{atmos}}{}
Short-wave radiative forcing of the ocean.
\begin{gather}
\begin{split}W_{i,j} = (\eta_i, F_j)\end{split}\notag
\end{gather}
\begin{notice}{note}{Note:}
atmospheric s tensor must be computed before calling
this function !
\end{notice}

\end{fulllineitems}


\end{fulllineitems}

\index{ocean\_wavenum (class in inprod\_analytic)}

\begin{fulllineitems}
\phantomsection\label{rstfiles/inprod_analytic:inprod_analytic.ocean_wavenum}\pysiglinewithargsret{\strong{class }\code{inprod\_analytic.}\bfcode{ocean\_wavenum}}{\emph{P}, \emph{H}, \emph{Nx}, \emph{Ny}}{}
Class to define ocean wavenumbers

Attributes :
\begin{itemize}
\item {} 
P (int)

\item {} 
H (int)

\item {} 
Nx (int)

\item {} 
Ny (int)

\end{itemize}

\end{fulllineitems}



\section{Tensor computation module}
\label{rstfiles/aotensor:tensor-computation-module}\label{rstfiles/aotensor::doc}
The equation tensor for the coupled ocean-atmosphere model
with temperature which allows for an extensible set of modes
in the ocean and in the atmosphere.

\begin{notice}{note}{Note:}
These are calculated using the analytical expressions from De     Cruz, L., Demaeyer, J. and Vannitsem, S.: A modular arbitrary-order     ocean-atmosphere model: MAOOAM v1.0, Geosci. Model Dev. Discuss.            And the \href{https://github.com/Climdyn/MAOOAM}{Fortran Code}
\end{notice}

\begin{notice}{note}{Note:}
The python code is available here :     aotensor.py .
\end{notice}
\begin{quote}\begin{description}
\item[{Example}] \leavevmode
\end{description}\end{quote}

\begin{Verbatim}[commandchars=\\\{\}]
\PYG{g+gp}{\PYGZgt{}\PYGZgt{}\PYGZgt{} }\PYG{n}{aotensor}\PYG{p}{,} \PYG{n}{Li}\PYG{p}{,} \PYG{n}{Lj}\PYG{p}{,} \PYG{n}{Lk}\PYG{p}{,} \PYG{n}{Lv} \PYG{o}{=}\PYG{n}{aotensor}\PYG{o}{.}\PYG{n}{init\PYGZus{}aotensor}\PYG{p}{(}\PYG{p}{)}
\end{Verbatim}


\subsection{Help Functions}
\label{rstfiles/aotensor:help-functions}
There are ndim coordinates that correspond to 4 physical quantities.
These functions help to have the i-th coordinates of a quantity.
\begin{itemize}
\item {} 
psi(i) -\textgreater{} i

\item {} 
theta(i) -\textgreater{} i + natm

\item {} 
A(i) -\textgreater{} i + 2*natm

\item {} 
T(i) -\textgreater{} i + 2*natm + noc

\item {} 
kdelta(i,j) -\textgreater{} (i==j)

\end{itemize}


\subsection{Global variables}
\label{rstfiles/aotensor:global-variables}\begin{itemize}
\item {} 
real\_eps = 2.2204460492503131e-16

\item {} 
t=np.zeros( ((ndim+1),(ndim+1),(ndim+1)),dtype=float)

\end{itemize}


\subsection{Dependencies}
\label{rstfiles/aotensor:dependencies}
\begin{Verbatim}[commandchars=\\\{\}]
\PYG{g+gp}{\PYGZgt{}\PYGZgt{}\PYGZgt{} }\PYG{k+kn}{from} \PYG{n+nn}{params\PYGZus{}maooam} \PYG{k+kn}{import} \PYG{o}{*}
\PYG{g+gp}{\PYGZgt{}\PYGZgt{}\PYGZgt{} }\PYG{k+kn}{from} \PYG{n+nn}{inprod\PYGZus{}analytic} \PYG{k+kn}{import} \PYG{o}{*}
\PYG{g+gp}{\PYGZgt{}\PYGZgt{}\PYGZgt{} }\PYG{k+kn}{from} \PYG{n+nn}{scipy.sparse} \PYG{k+kn}{import} \PYG{n}{dok\PYGZus{}matrix}
\PYG{g+gp}{\PYGZgt{}\PYGZgt{}\PYGZgt{} }\PYG{k+kn}{from} \PYG{n+nn}{scipy.sparse} \PYG{k+kn}{import} \PYG{n}{csr\PYGZus{}matrix}
\PYG{g+gp}{\PYGZgt{}\PYGZgt{}\PYGZgt{} }\PYG{k+kn}{import} \PYG{n+nn}{os}
\end{Verbatim}


\subsection{Functions}
\label{rstfiles/aotensor:functions}\begin{itemize}
\item {} 
compute\_aotensor

\item {} 
coeff(i, j, k, v)

\item {} 
simplify

\item {} 
init\_aotensor

\end{itemize}


\begin{fulllineitems}
\pysiglinewithargsret{\code{aotensor.}\bfcode{coeff}}{\emph{i}, \emph{j}, \emph{k}, \emph{v}}{}~\begin{description}
\item[{Affects v for \(t_{i,j,k}\) making that tensor{[}i{]} upper triangular.}] \leavevmode
Used in compute\_aotensor.

\end{description}
\begin{quote}\begin{description}
\item[{Parameters}] \leavevmode\begin{itemize}
\item {} 
\textbf{\texttt{i}} (\emph{\texttt{int in {[}1,37{]}}}) -- first coordinates

\item {} 
\textbf{\texttt{j}} (\emph{\texttt{int in {[}1,37{]}}}) -- second coodinates

\item {} 
\textbf{\texttt{k}} (\emph{\texttt{int in {[}1,37{]}}}) -- third coordinates

\item {} 
\textbf{\texttt{v}} (\href{http://docs.python.org/library/functions.html\#float}{\emph{\texttt{float}}}) -- value

\end{itemize}

\item[{Returns}] \leavevmode
change the global tensor

\item[{Return type}] \leavevmode
void

\item[{Example}] \leavevmode
\end{description}\end{quote}

\begin{Verbatim}[commandchars=\\\{\}]
\PYG{g+gp}{\PYGZgt{}\PYGZgt{}\PYGZgt{} }\PYG{n}{coeff}\PYG{p}{(}\PYG{n}{i}\PYG{p}{,} \PYG{n}{j}\PYG{p}{,} \PYG{n}{k}\PYG{p}{,} \PYG{n}{v}\PYG{p}{)}
\end{Verbatim}

\end{fulllineitems}



\begin{fulllineitems}
\pysiglinewithargsret{\code{aotensor.}\bfcode{compute\_aotensor}}{}{}
Computes the three-dimensional tensor t

Takes the inner products of inprod\_analytic         and computes the tensor
\begin{quote}\begin{description}
\item[{Parameters}] \leavevmode
\textbf{\texttt{t}} (\emph{\texttt{array((37,37,37),float)}}) -- tensor t is a global variable of aotensor

\item[{Returns}] \leavevmode
change the global tensor

\item[{Return type}] \leavevmode
void

\item[{Example}] \leavevmode
\end{description}\end{quote}

\begin{Verbatim}[commandchars=\\\{\}]
\PYG{g+gp}{\PYGZgt{}\PYGZgt{}\PYGZgt{} }\PYG{n}{compute\PYGZus{}aotensor}\PYG{p}{(}\PYG{p}{)}
\end{Verbatim}

\begin{notice}{warning}{Warning:}
Needs the global variable aotensor and the global inner         products to be initialized.
\end{notice}

\end{fulllineitems}



\begin{fulllineitems}
\pysiglinewithargsret{\code{aotensor.}\bfcode{init\_aotensor}}{}{}
Initialize the tensor.
\begin{quote}\begin{description}
\item[{Returns}] \leavevmode
aotensor, Li, Lj, Lk, Lv

\item[{Example}] \leavevmode
\end{description}\end{quote}

\begin{Verbatim}[commandchars=\\\{\}]
\PYG{g+gp}{\PYGZgt{}\PYGZgt{}\PYGZgt{} }\PYG{n}{aotensor}\PYG{p}{,} \PYG{n}{Li}\PYG{p}{,} \PYG{n}{Lj}\PYG{p}{,} \PYG{n}{Lk}\PYG{p}{,} \PYG{n}{Lv} \PYG{o}{=} \PYG{n}{init\PYGZus{}aotensor}\PYG{p}{(}\PYG{p}{)}
\end{Verbatim}

\end{fulllineitems}



\begin{fulllineitems}
\pysiglinewithargsret{\code{aotensor.}\bfcode{simplify}}{}{}
Make sure that tensor{[}i{]} is upper triangular.
To do after compute\_aotensor().
\begin{quote}\begin{description}
\item[{Parameters}] \leavevmode
\textbf{\texttt{t}} (\emph{\texttt{array((ndim+1,ndim+1,ndim+1),float)}}) -- tensor t is a global variable of aotensor

\item[{Returns}] \leavevmode
change the global tensor

\item[{Return type}] \leavevmode
void

\item[{Example}] \leavevmode
\end{description}\end{quote}

\begin{Verbatim}[commandchars=\\\{\}]
\PYG{g+gp}{\PYGZgt{}\PYGZgt{}\PYGZgt{} }\PYG{n}{simplify}\PYG{p}{(}\PYG{p}{)}
\end{Verbatim}

\end{fulllineitems}

\phantomsection\label{rstfiles/integrator:module-integrator}\index{integrator (module)}

\section{Integration module}
\label{rstfiles/integrator::doc}\label{rstfiles/integrator:integration-module}
This module actually contains the Heun algorithm routines.

\begin{notice}{note}{Note:}
The python code is available here :     integrator.py .
\end{notice}
\begin{quote}\begin{description}
\item[{Example}] \leavevmode
\end{description}\end{quote}

\begin{Verbatim}[commandchars=\\\{\}]
\PYG{g+gp}{\PYGZgt{}\PYGZgt{}\PYGZgt{} }\PYG{k+kn}{from} \PYG{n+nn}{integrator} \PYG{k+kn}{import} \PYG{n}{step}
\PYG{g+gp}{\PYGZgt{}\PYGZgt{}\PYGZgt{} }\PYG{n}{step}\PYG{p}{(}\PYG{n}{y}\PYG{p}{,}\PYG{n}{t}\PYG{p}{,}\PYG{n}{dt}\PYG{p}{)}
\end{Verbatim}


\subsection{Global variables}
\label{rstfiles/integrator:global-variables}\begin{itemize}
\item {} 
\textbf{aotensor} tensor with the format (int i, int j, int k, float v) in list

\item {} 
\textbf{Li} first list of index of tensor

\item {} 
\textbf{Lj} second list of index of tensor

\item {} 
\textbf{Lk} third list of index of tensor

\item {} 
\textbf{Lv} list of tensor values

\end{itemize}


\subsection{Dependencies}
\label{rstfiles/integrator:dependencies}
\begin{Verbatim}[commandchars=\\\{\}]
\PYG{g+gp}{\PYGZgt{}\PYGZgt{}\PYGZgt{} }\PYG{k+kn}{import} \PYG{n+nn}{numpy} \PYG{k+kn}{as} \PYG{n+nn}{np}
\PYG{g+gp}{\PYGZgt{}\PYGZgt{}\PYGZgt{} }\PYG{k+kn}{from} \PYG{n+nn}{params\PYGZus{}maooam} \PYG{k+kn}{import} \PYG{n}{ndim}\PYG{p}{,}\PYG{n}{f2py}
\PYG{g+gp}{\PYGZgt{}\PYGZgt{}\PYGZgt{} }\PYG{k+kn}{import} \PYG{n+nn}{aotensor} \PYG{k+kn}{as} \PYG{n+nn}{aotensor\PYGZus{}mod}
\PYG{g+gp}{\PYGZgt{}\PYGZgt{}\PYGZgt{} }\PYG{k}{if} \PYG{n}{f2py}\PYG{p}{:}
\PYG{g+gp}{\PYGZgt{}\PYGZgt{}\PYGZgt{} }    \PYG{k+kn}{import} \PYG{n+nn}{sparse\PYGZus{}mult} \PYG{k+kn}{as} \PYG{n+nn}{mult}
\PYG{g+gp}{\PYGZgt{}\PYGZgt{}\PYGZgt{} }    \PYG{n}{sparse\PYGZus{}mul3\PYGZus{}f2py} \PYG{o}{=} \PYG{n}{mult}\PYG{o}{.}\PYG{n}{sparse\PYGZus{}mult}\PYG{o}{.}\PYG{n}{sparse\PYGZus{}mul3}
\end{Verbatim}


\subsection{Functions}
\label{rstfiles/integrator:functions}\begin{itemize}
\item {} 
sparse\_mul3

\item {} 
tendencies

\item {} 
step

\end{itemize}
\index{sparse\_mul3() (in module integrator)}

\begin{fulllineitems}
\phantomsection\label{rstfiles/integrator:integrator.sparse_mul3}\pysiglinewithargsret{\code{integrator.}\bfcode{sparse\_mul3}}{\emph{arr}}{}
Calculate for each i the sums on j,k of the product
\begin{gather}
\begin{split}tensor(i,j,k)* arr(j) * arr(k)\end{split}\notag
\end{gather}
\begin{notice}{note}{Note:}
Python-only function
\end{notice}

\end{fulllineitems}

\index{sparse\_mul3\_py() (in module integrator)}

\begin{fulllineitems}
\phantomsection\label{rstfiles/integrator:integrator.sparse_mul3_py}\pysiglinewithargsret{\code{integrator.}\bfcode{sparse\_mul3\_py}}{\emph{arr}}{}
Calculate for each i the sums on j,k of the product
\begin{gather}
\begin{split}tensor(i,j,k)* arr(j) * arr(k)\end{split}\notag
\end{gather}
\begin{notice}{note}{Note:}
Python-only function
\end{notice}

\end{fulllineitems}

\index{step() (in module integrator)}

\begin{fulllineitems}
\phantomsection\label{rstfiles/integrator:integrator.step}\pysiglinewithargsret{\code{integrator.}\bfcode{step}}{\emph{y}, \emph{t}, \emph{dt}}{}
RK2 method integration

\end{fulllineitems}

\index{tendencies() (in module integrator)}

\begin{fulllineitems}
\phantomsection\label{rstfiles/integrator:integrator.tendencies}\pysiglinewithargsret{\code{integrator.}\bfcode{tendencies}}{\emph{y}}{}
Calculate the tendencies thanks to the product of the tensor and     the vector y

\end{fulllineitems}

\phantomsection\label{rstfiles/maooam:module-maooam}\index{maooam (module)}

\section{Principal module}
\label{rstfiles/maooam::doc}\label{rstfiles/maooam:principal-module}
Python implementation of the Modular Arbitrary-Order Ocean-Atmosphere Model MAOOAM

\begin{notice}{note}{Note:}
The python code is available here :     maooam.py .
\end{notice}
\begin{quote}\begin{description}
\item[{Example}] \leavevmode
\end{description}\end{quote}

\begin{Verbatim}[commandchars=\\\{\}]
\PYG{g+gp}{\PYGZgt{}\PYGZgt{}\PYGZgt{} }\PYG{k+kn}{from} \PYG{n+nn}{maooam} \PYG{k+kn}{import} \PYG{o}{*}
\end{Verbatim}


\subsection{Global variable}
\label{rstfiles/maooam:global-variable}\begin{itemize}
\item {} 
\textbf{ic.X0} : initial conditions

\item {} 
\textbf{X} : live step vector

\item {} 
\textbf{t} : time

\item {} 
\textbf{t\_trans}, \textbf{t\_run} : respectively transient and running time

\item {} 
\textbf{dt} : step time

\item {} 
\textbf{tw} : step time for writing on evol\_field.dat

\end{itemize}


\subsection{Dependencies}
\label{rstfiles/maooam:dependencies}
\begin{Verbatim}[commandchars=\\\{\}]
\PYG{g+gp}{\PYGZgt{}\PYGZgt{}\PYGZgt{} }\PYG{k+kn}{import} \PYG{n+nn}{numpy} \PYG{k+kn}{as} \PYG{n+nn}{np}
\PYG{g+gp}{\PYGZgt{}\PYGZgt{}\PYGZgt{} }\PYG{k+kn}{import} \PYG{n+nn}{params\PYGZus{}maooam}
\PYG{g+gp}{\PYGZgt{}\PYGZgt{}\PYGZgt{} }\PYG{k+kn}{from} \PYG{n+nn}{params\PYGZus{}maooam} \PYG{k+kn}{import} \PYG{n}{ndim}\PYG{p}{,}\PYG{n}{tw}\PYG{p}{,}\PYG{n}{t\PYGZus{}run}\PYG{p}{,}\PYG{n}{t\PYGZus{}trans}\PYG{p}{,}\PYG{n}{dt}
\PYG{g+gp}{\PYGZgt{}\PYGZgt{}\PYGZgt{} }\PYG{k+kn}{import} \PYG{n+nn}{aotensor}
\PYG{g+gp}{\PYGZgt{}\PYGZgt{}\PYGZgt{} }\PYG{k+kn}{import} \PYG{n+nn}{time}
\PYG{g+gp}{\PYGZgt{}\PYGZgt{}\PYGZgt{} }\PYG{k+kn}{import} \PYG{n+nn}{ic\PYGZus{}def}
\PYG{g+gp}{\PYGZgt{}\PYGZgt{}\PYGZgt{} }\PYG{k+kn}{import} \PYG{n+nn}{ic}
\PYG{g+gp}{\PYGZgt{}\PYGZgt{}\PYGZgt{} }\PYG{k+kn}{import} \PYG{n+nn}{sys}
\end{Verbatim}
\index{bcolors (class in maooam)}

\begin{fulllineitems}
\phantomsection\label{rstfiles/maooam:maooam.bcolors}\pysigline{\strong{class }\code{maooam.}\bfcode{bcolors}}
to color the instructions in the console

\end{fulllineitems}



\chapter{Contributors}
\label{index:contributors}
Maxime Tondeur, Jonathan Demaeyer


\chapter{License}
\label{index:license}
© 2017 Maxime Tondeur and Jonathan Demaeyer

See LICENSE.txt  for license information.


\chapter{Indices and tables}
\label{index:indices-and-tables}\begin{itemize}
\item {} 
\DUspan{xref,std,std-ref}{genindex}

\item {} 
\DUspan{xref,std,std-ref}{modindex}

\item {} 
\DUspan{xref,std,std-ref}{search}

\end{itemize}


\renewcommand{\indexname}{Python Module Index}
\begin{theindex}
\def\bigletter#1{{\Large\sffamily#1}\nopagebreak\vspace{1mm}}
\bigletter{i}
\item {\texttt{ic}}, \pageref{rstfiles/ic:module-ic}
\item {\texttt{ic\_def}}, \pageref{rstfiles/ic_def:module-ic_def}
\item {\texttt{inprod\_analytic}}, \pageref{rstfiles/inprod_analytic:module-inprod_analytic}
\item {\texttt{integrator}}, \pageref{rstfiles/integrator:module-integrator}
\indexspace
\bigletter{m}
\item {\texttt{maooam}}, \pageref{rstfiles/maooam:module-maooam}
\indexspace
\bigletter{p}
\item {\texttt{params\_maooam}}, \pageref{rstfiles/params_maooam:module-params_maooam}
\end{theindex}

\renewcommand{\indexname}{Index}
\printindex
\end{document}
